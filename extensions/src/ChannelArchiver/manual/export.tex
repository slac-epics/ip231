\section{\INDEX{ArchiveExport}}
ArchiveExport is a command-line tool for local tests, i.e.\ it does
\emph{not} connect to the archive data server, but instead requires that
you have read access to the index and data files.
It is mostly meant for testing.

When invoked without valid arguments, it will display a command
description.

\begin{lstlisting}[frame=none,keywordstyle=\sffamily]
USAGE: ArchiveExport [Options] <index file> {channel}
 
Options:
  -verbose               Verbose mode
  -list                  List all channels
  -info                  Time-range info on channels
  -start <time>          Format:
                      "mm/dd/yyyy[ hh:mm:ss[.nano-secs]]"
  -end <time>            (exclusive)
  -text                  Include text column for
                         status/severity
  -match <reg. exp.>     Channel name pattern
  -interpolate <seconds> interpolate values
  -output <file>         output to file
  -gnuplot               generate GNUPlot command file
  -Gnuplot               generate GNUPlot output
                         for Image
\end{lstlisting}

\noindent ArchiveExport produces spreadsheet-type output in
TAB-separated ASCII text, suitable for import into most spreadsheet
programs for further analysis. Per default, ArchiveExport uses
Staircase Interpolation to map the data for the requested channels
onto matching time stamps, but one can select Linear Interpolation via
the ``-interpolate'' option. When GNUPlot output is selected, the
Plot-Binning method is used, where the bin size is determined by the
``-interpolate'' argument. See section \ref{sec:timestampcorr} on
page~\pageref{sec:timestampcorr} for details.

Assuming that your current working directory contains an
archive index file that is aptly named ``index'', the following
invocation will generate a spreadsheet file ``data.txt'' with the data
of all channels that match the pattern ``IOC'' for the date of January
27, 2003:

\begin{lstlisting}[frame=none,keywordstyle=\sffamily]
ArchiveExport index -m IOC \
              -s "01/27/2003" \
              -e "01/28/2003" >data.txt
\end{lstlisting}

\noindent To plot this in OpenOffice, you could create a new
spreadsheet, then use the menu item Insert/Sheet/FromFile, select the
file ``data.txt'' and configure the text import dialog to use
``Separated by Tab''. You will notice that even though the original
text file contains time stamps with nano-second resolution, for
example ``01/27/2003 23:57:25.579346999'', the spreadsheet program
might use a default representation of e.g. 
``01/27/03 23:57 pm''.
In order to see the full time stamp detail, one needs to reformat
those spreadsheet columns with a user-defined format like
``MM/DD/YYYY HH:MM:SS.000000000''.
If you use Microsoft Excel, you might be limited to a format with
millisecond resolution: ``MM/DD/YYYY HH:MM:SS.000''.
For graphing the data, the most suitable option is often an
``X-Y-Graph'', using the first row for labels, with the data series
taken from the columns.

The following call sequence will generate a GNUPlot data file ``data''
together with a GNUPlot command file ``data.plt'' and execute it
within GNUPlot:
\begin{lstlisting}[frame=none,keywordstyle=\sffamily]
ArchiveExport index -m IOC \
              -s "01/27/2003" \
              -e "01/28/2003" -o data -gnu
gnuplot
        G N U P L O T
        Version 3.7 patchlevel 3
....
gnuplot> load 'data.plt'
\end{lstlisting}
