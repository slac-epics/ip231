\section{Index Tool} \label{sec:indextool}
The \INDEX{ArchiveIndexTool} is used to create \INDEX{Master Indices}
by combining multiple indices into a new one.  When invoked without
valid arguments, it will display a command description similar to
this:

\begin{lstlisting}[frame=none,keywordstyle=\sffamily]
USAGE: ArchiveIndexTool [Options] <archive list file> \
                                         <output index> 
Options:
  -help             Show Help
  -M <3-100>        RTree M value
  -verbose <level>  Show more info
\end{lstlisting}

\noindent The archive list file lists all the sub archives,
that is the paths to each sub-archive's index file. It needs to be an
XML file conforming to the DTD in listing \ref{lst:indexconfigdtd}
(see section \ref{sec:dtdfiles} on DTD file installation).
Listing \ref{lst:indexconfigex} provides an example.

\lstinputlisting[float=htb,keywordstyle=\sffamily,caption={XML DTD for
    the Archive Index Tool Configuration},label=lst:indexconfigdtd]{../IndexTool/indexconfig.dtd}

\lstinputlisting[float=htb,keywordstyle=\sffamily,caption={Example
    Archive Index Tool Configuration},label=lst:indexconfigex]{../IndexTool/indexconfig.xml}

We refer to chapter \ref{ch:examplesetup} for an example of how to use
the ArchiveIndexTool in collaboration with the other Channel Archiver
tools.

\subsection{make\_indexconfig.pl} \label{sec:makeindexconfig}
As an aid to creating configuration files for the ArchiveIndexTool,
you can use the perl script ``make\_indexconfig.pl'' that converts a
list of index files into the appropriately formatted XML:
\begin{lstlisting}[frame=none,keywordstyle=\sffamily]

USAGE: make_indexconfig [-d DTD] index { index }
 
This tool generates a configuration for the
ArchiveIndexTool based on a DTD and a list
of index files provided via the command line.
\end{lstlisting}

\subsection{Internals}
The Index Tool allows creation of a master index
that covers more than one sub archive. For example, we can create
configuration files for the index tool to combine several vacuum
sub-archives into one, then do the same for cooling data:
\begin{lstlisting}[frame=none,keywordstyle=\sffamily]
cd /data/vacuum
make_indexconfig.pl 2004/02_19/index 2004/02_20/index \
     >indexconfig.xml
cd /data/cooling
make_indexconfig.pl 2004/02_19/index 2004/02_20/index  \
     > indexconfig.xml
\end{lstlisting}

\noindent After running ArchiveIndexTool in /data/vacuum and /data/cooling,
we will have two new indices. One refers to all the vacuum data, the
other to all the cooling data:
\begin{lstlisting}[frame=none,keywordstyle=\sffamily]
/data/vacuum/index
/data/cooling/index
\end{lstlisting}
\noindent Note that these are only index files. There are no new data
files because the new ``master'' index files will point to data blocks
in the existing data files, e.g. the one under /data/vacuum/2004/02\_19. 
It is also important to remember that the master index files include the
paths to the data files as instructed in the indexconfig.xml files.
According to the previous example,  
/data/vacuum/index was created from
/data/vacuum/indexconfig.xml which included the relative path
``2004/02\_19/index''. The vacuum master index will therefore point to
data files with a relative path like ``2004/02\_19/20040219''.
Whenever we use ``/data/vacuum/index'', the retrieval tools will prepend
the path to the index, ``/data/vacuum'', to the relative data 
file path found in the index, for example ``2004/02\_19/20040219'', and thus
find the data under its full path of
 e.g.\ ``/data/vacuum/2004/02\_19/20040219''.
We cannot move ``/data/vacuum/index'' to another location like
``/tmp/index''. The retrieval tools would then try to access
``/tmp/2004/02\_19/20040219'' and fail.

Having said that, it \emph{is} possible to generate master indices
that use the full, absolute paths to their data files by simply listing
the full paths to the sub-archives in indexconfig.xml. This is,
however, not recommended because it will increase the size of the
index files simply because the full path names are longer than the
relative paths. For the same reason it is advisable to use short path
names: When an index file points to many data blocks in many data
files, it makes quite some difference if you used a short-named
directory tree with paths like ``/data/vac/...'' as opposed to
``/user/data/channel-archiver/data/subsystems/vacuum-system/...''.

As a second step, we can further combine the master indices for vacuum
and cooling data into one index that covers all out data. By creating
``/data/indexconfig.xml'' in which we list ``vacuum/index'' and
``cooling/index'', and running the ArchiveIndexTool in
``/data'', we create ``/data/index'' which points to all our data.
Alternatively, we could have skipped the intermediate indices for
vacuum and cooling and created ``/data/indexconfig.xml'' from the
beginning like this:
\begin{lstlisting}[frame=none,keywordstyle=\sffamily]
cd data
make_indexconfig.pl */2004/*/index  \
                    >indexconfig.xml
ArchiveIndexTool -v1 indexconfig index
\end{lstlisting}

\noindent In any case we end up with ``/data/index'' as an index for
all our vacuum and cooling data.
