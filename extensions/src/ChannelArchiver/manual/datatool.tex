\section{Data Tool} \label{sec:datatool}

The \INDEX{ArchiveDataTool} allows investigation of data files as well as
conversion from old directory-file based archives into ones that
utilize an index file.
It also allows some basic \INDEX{data management} as described in the next
sections.

\begin{lstlisting}[frame=none,keywordstyle=\sffamily] 
USAGE: ArchiveDataTool [Options] <index-file>
 
Options:
  -help                     Show help
  -verbose <level>          Show more info
  -list                     List channel name info
  -dir2index <dir. file>    Convert old directory file
                            to index
  -index2dir <dir. file>    Convert index to old
                            directory file
  -M <1-100>                RTree M value
  -blocks                   List data blocks of a channel
  -Blocks                   List all data blocks
  -dotindex <dot filename>  Dump contents of RTree index
                            into dot file
  -channel <name>           Channel name
  -hashinfo                 Show Hash table info
  -seek <time>              Perform seek test
  -test                     Perform consistency tests
\end{lstlisting}

\subsection{\INDEX{Delete Channels, Data}} \label{sec:deleteDataFromArchive}
Given one index file and its associated data files, you cannot remove
a channel, the data for a channel, or the data for a certain time
range from that sub-archive. In principle one can imagine deleting a
channel's information from the index file, but that simply turns its
data blocks inside the data files into orphans. It does not release
the disk space occupied by the channel's samples, so why bother?
Similarly, removing the index entries of one channel for a specific
time range would not free up the associated disk space inside the data
files.

Looking at data files with names that indicate dates, for example
``20051210'', ``20051211'' and ``20051212'', one might be led to
believe that deleting the file ``20051211'' frees up disk space and
deletes the data for that day in December, 2005.  It \emph{will} free
up disk space, alright. But the file name ``20051211'' only indicates
the creation date of that file.  It will contain samples of channels
from that date on, until data buffers inside the file get filled.  So
it can contain data up to Christmas 2005 and further.  For other
channels, data for Dec.\ 11 and 12 will actually be in the file ``20051210'',
because a data buffer in there was still being filled on Dec.\ 12.
The only way of knowing what channels and time ranges reside in which
file is to look at all the data blocks for all the channels
(``-blocks'' option).  Furthermore, since the data blocks are interlinked,
deleting one data file in a sub-archive might confuse the retrieval
routines.

\NOTE The short summary is that one should under no circumstances try
to directly modify or delete index and associated data files of a sub-archive.

What can you do? Delete the whole set of index and associated data
files! This is one reason for creating daily or weekly sub-archives,
so you can move, copy and delete them without affecting other
sub-archives.

One can also use the ``-copy'' option to copy a time range into a
\emph{new} sub-archive, and then delete the original.
After deleting a sub-archive or replacing it with a copied-out time
slice, you have to recreate indices that referred to that sub-archive.

\subsection{\INDEX{Combine Sub-Archives}}
The generation of daily or weekly sub-archives reduces the amount of data
endangered by ArchiveEngine crashes. In the long run, however, it is often
advisable to combine the daily or weekly sub-archives into bigger ones, for
example monthly. The smaller number of sub-archives is easier to
handle when it comes to backups. Is also provides slightly better
\INDEX{retrieval times}. 
In the following example, we assume that it's February 2004 and we want
to combine daily vacuum sub-archives into one for the month of January
2004.
\begin{lstlisting}[frame=none,keywordstyle=\sffamily]
cd vacuum/2004
mkdir 01_xx
ArchiveDataTool -copy 01_xx/index 01_01/index \
    -e "01/02/2004 02:00:00"
ArchiveDataTool -copy 01_xx/index 01_02/index \
    -s "01/02/2004 02:00:00" -e "01/03/2004 02:00:00"
ArchiveDataTool -copy 01_xx/index 01_03/index \
    -s "01/03/2004 02:00:00" -e "01/04/2004 02:00:00"
...
\end{lstlisting}
\noindent Note that we assume a daily restart at 02:00 and thus we
force the ArchiveDataTool to only copy values from the time range
where we expect the sub-archives to have data. This practice somewhat
helps us to remove samples with wrong time stamps that result from
Channel Access servers with ill-configured clocks.

There is a perl command \INDEX{make\_compress\_script.pl} that aids in the
creation of a shell script for the ArchiveDataTool, but you need to
review it carefully before invokation.
Depending on your situation, monthly archives might either
be too big to fit on a CD-ROM or ridiculously small, in which case you
should try weekly, bi-weekly, quarterly or other time ranges for your sub-archives.

After successfully combining the daily sub-archives into a monthly one,
you need to move that dayly data out of the way and finally
rebuild indices that use the data.

\subsection{\INDEX{Reduce the Data Size}, \INDEX{Data File Repair}}
The simple ``copy'' of one sub-archive onto another one like this will
keep all channels and samples, but typically reduce the size of the
data files.  For daily sub-archives, it can be a reduction down to
1/10th of the original file sizes.

\begin{lstlisting}[frame=none,keywordstyle=\sffamily]
cd vacuum/2004
mkdir 01_01_copy
ArchiveDataTool  01_01/index -copy 01_01_copy/index
\end{lstlisting}

\noindent The reason lies in the fact that the data files contain
\INDEX{data buffers}, allocated by the engine without knowing how many
samples to expect. So initially, a small buffer is allocated. When
full, a new buffer of twice the original size is allocated and so on,
up to a maximum buffer size. For engines that run briefly, for example
only one day, many of these buffers are partially filled when the
sub-archive is closed.  When copied, the archive data tool can count
the available samples before allocating data buffers for the copy, and
thereby reducing the number of buffers and also avoiding any unused
buffer space.

In addition, the copy process will skip data errors, omitting samples
with time stamps that go backwards in time,
or data buffers with invalid pointers to control information.
It will \emph{not} perform any real \INDEX{data file repair} in the sense
of magically assiging correct time stamps or control information, so
the affected samples are simply lost, but the resuling copy of the
original sub-archive should no longer result in any error messages on retrieval.

