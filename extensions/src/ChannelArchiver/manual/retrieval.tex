\chapter{Data Retrieval}  \label{chap:retrieval}
Data retrieval requirements can cover a wide range.  One person might
be interested in the temperature of a water tank during the last
night. For this, it is probably sufficient to retrieve the raw data
for the respective channel and plot it.  If, on the other hand, we
want to look at the same temperature for the last 3 month, the raw
data will amount to too many samples and some sort of data reduction
or interpolation is helpful.  We already mentioned the problems of
time stamp correlation that arise when comparing different channels in
section \ref{sec:timestampcorr}.

The Channel Archiver toolset includes some generic tools that can be
used ``as is''. While those try to cover many data retrieval
requirements, certain requests can only be handled in customized data
mining programs (which might be e.g.\ perl scripts). For this, the
archiver offers a network data server. In short, these are your
fundamental options:

\begin{itemize}
\item Java Archive Client\\
      This is meant to be \emph{the} data client. Use it to browse the
      available data, generate plots, export data to spreadsheets,
      from any computer on the network, by accessing the data server.
\item ArchiveExport\\
      A command-line tool. Less convenient to use, requires direct
      access to the data files. Use this when the Java Archive Client
      or network data server are not available.
\item Archive Data Server\\
      Serves data to the Java Archive Client. In addition, you can
      access the documented XML-RPC protocol of the data server from
      most programming languages. Use this method for customized data
      mining programs.
\item ``Storage'' Library\\
      A C++ library for accessing local data files. Use for
      specialized C++ code.
\end{itemize}