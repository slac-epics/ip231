\section{\INDEX{Data Server}} \label{sec:dataserver}
The archiver toolset includes a network data server.
By running this data server on a computer that has
physical access to your archived data, be it because the data resides
on a local disk or an NFS-mapped volume, other machines
on the network can get read-access to your data.

\begin{figure}[htb]
\begin{center}
\InsertImage{width=0.8\textwidth}{dataserver}
\end{center}
\caption{\label{fig:dataserver}Data Server, refer to text.}
\end{figure}

\noindent The data server is hosted by a web server, using the XML-RPC
protocol to serve the data. This means that software running on
disparate operating systems, running in different environments can
access your data over the Internet via a URL. As an example, your data
server might be a Linux machine on a subnet behind a firewall. After
you configure the firewall to pass HTTP requests, any Linux, Win32,
Macintosh computer both inside or outside of the firewall can access
the data from within perl, python or tcl scripts, programs written in
C, C++ or Java, actually pretty much any programming language. As
illustrated in fig.~\ref{fig:dataserver}, the client program sends its
requests to a web server, which forwards it to the data server that is
running as a CGI tool under the web server. The dataserver accesses
the relevant archives --- you determine which ones are available via a
configuration file --- and returns the data though the web server to
the client program.  You can configure access security via e.g.\ the
Apache web server configuration.

\NOTE The fact that the data server is hosted by a web server,
accessible via a URL, does \emph{not} imply that you can use any web browser
to retrieve data. You have to use the XML remote procedure call
protocol described in section \ref{sec:xmlprotocol} on page
\pageref{sec:xmlprotocol}. XML-RPC handles the network connections,
data type conversions, and XML-RPC libraries are available for most
programming languages.
We provide the Java Archive Client (see section \ref{sec:javaclient})
as a generic, interactive data client. 

\subsection{Installation} % ---------------------------------------------
After successful compilation in ChannelArchiver/XMLRPCServer, you should
have a program
``XMLRPCServer/O.\$EPICS\_HOST\_ARCH/\INDEX{ArchiveDataServer}''.
You have to install it as a CGI tool under a web server, passing
an environment variable ``SERVERCONFIG'' to it which points
to a configuration file. This means

\begin{itemize}
\item You need a web server like \INDEX{Apache} running on a computer
      with access to the archive data.
\item You need to know how to start, stop and configure that
      web server.
\item You need to know how to install, configure and possibly
      troubleshoot \INDEX{CGI} tools under that web server.
\item You need to create a configuration file for the ArchiveDataServer.
\end{itemize}

\noindent With all but the last point you are very much on your own, because
this manual cannot even try to describe all the possible configurations
and errors.
What follows is only an example setup for the Apache Web Server under Linux
and Mac OS X. Your web server, even if it is also a version of Apache,
is likely to be quite different.

\begin{enumerate}
\item Locate your web server configuration file. This is is often 
  a variant of ``/etc/httpd/conf/httpd.conf''. For Mandrake 10, check
  ``/etc/httpd/conf/commonhttpd.conf'',
  for Mac OS X it's ``/etc/httpd/httpd.conf''.
\item After each change to the configuration, you will have to restart the
 web server. This is often done via 
``/etc/rc.d/initd/httpd restart''
or
``/usr/sbin/apachectl restart''.
\item Create a new web directory for the archiver with a CGI
   sub-directory. This is typically done under /var/www/html,
   except for Mac OS~X, where you would use /Library/WebServer/Documents.
   Check the ``DocumentRoot'' variable in your web server configuration file
   for the correct location and adjust the following accordingly.
\begin{lstlisting}[keywordstyle=\sffamily]
mkdir /var/www/html/archive
mkdir /var/www/html/archive/cgi
\end{lstlisting}
   Change the permissions of those directories to your liking.
   Usually, ``everybody'' needs read and execution access, because
   the web server will run CGI programs as a low-priviledged user. 
   Our main interest here is the CGI''cgi'' subdirectory.
   You can use the ``archive'' directory to store e.g.\ the dtd files
   or web pages with user information that relate to the archive setup
   at your site.
\item Copy the ArchiveDataServer binary into the cgi directory as
  ``ArchiveDataServer.cgi''.
   In this example, both the ``cgi'' directory and the ``.cgi''
   extension of ``ArchiveDataServer.cgi'' are important,
   because they identify the binary as a CGI program.
\item Assert that the web server can execute the ArchiveDataServer,
   that it recognizes it as a CGI tool, by
   adding the following to the Apache config file:
\begin{lstlisting}[keywordstyle=\sffamily]
# Check that environment variables are available,
# assert that this directive is not commented-out.
# Your web server might use a different module name
# or location, in my case it happened to be this:
LoadModule env_module  libexec/httpd/mod_env.so
AddModule mod_env.c

# Check that CGI is enabled, i.e. assert that
# these are not commented-out:
LoadModule cgi_module    libexec/httpd/mod_cgi.so
AddModule mod_cgi.c

# This tells Apache that ArchiveDataServer.cgi
# is a CGI program because of the .cgi extension:
AddHandler cgi-script .cgi

<Directory /var/www/html/archive>
   Order Allow,Deny
   Allow from All
</Directory>

# Allow cgi-scripts in the ..../cgi directory:
<Directory /var/www/html/archive/cgi>
   SetEnv EPICS_TS_MIN_WEST 300
   SetEnv LD_LIBRARY_PATH  /usr/local/lib:...
   SetEnv SERVERCONFIG \
    /var/www/html/archive/cgi/serverconfig.xml

   # I also like to enable perl-CGI, but that is
   # unrelated to the archiver
   PerlHandler Apache::Registry
   PerlSendHeader On

   # This directive enables CGI for this dir.:
   Options +ExecCGI
</Directory>
\end{lstlisting}
  The LD\_LIBRARY\_PATH needs to list all the directories that
  contain shared libraries which your ArchiveDataServer.cgi uses.
  In most cases, this includes the
  \begin{itemize}
  \item install location of the expat and XML-RPC
        libraries, often /usr/local/lib,
  \item ``lib'' subdirectories of EPICS base and EPICS extensions,
        something like /ade/epics/supTop/base/R3.14.6/lib/linux-x86:...
  \end{itemize}
  The SERVERCONFIG variable needs to point to your server configuration
  file, more about which next. The ExecCGI option is essential to
  allow CGI functionality. You can skip the Perl configuration for the
  data server.
\end{enumerate}

\subsection{Configuration}  % ---------------------------------------------
You need to prepare an XML-formatted configuration file for the
ArchiveDataServer that follows the DTD from listing
\ref{lst:serverconfigdtd} (see section \ref{sec:dtdfiles} on DTD file
installation). Note that the ArchiveDataServer might not verify your
configuration file, so you are strongly encouraged to use a tool like
'xmllint' on Linux to check your configuration against the
DTD. Listing \ref{lst:serverconfigex} shows one example 
\INDEX{serverconfig.xml} which lists two archives to be served. Client
programs will internally use the respective 'key' to access them.

\lstinputlisting[float=htb,keywordstyle=\sffamily,caption={XML DTD for the Data Server Configuration},label=lst:serverconfigdtd]{../XMLRPCServer/serverconfig.dtd}
\lstinputlisting[float=htb,keywordstyle=\sffamily,caption={Example Data Server Configuration},label=lst:serverconfigex]{../XMLRPCServer/serverconfig.xml}

\subsection{Testing, Debugging}  % ---------------------------------------------
Section \ref{sec:perlclient} describes a perl client to the network
data server. It can be used to see if all the archives you listed
in the configuration are actually accessible and so on.

In case that client tool gets nothing but errors, debugging of the
CGI ArchiveDataServer can be difficult, since one cannot easily
peek into the running (or failing to run) program when it is
launched by the web server.

The XMLRPCServer directory contains some self-tests in test.sh,
where the ArchiveDataServer is run without an actual web server.
If test.sh works, your ArchiveDataServer program is fundamentally
functional.

If it doesn't work from within your web server, try changing
your user ID to 'guest' or 'nobody', somebody similar to the user ID
used by the web server when it runs your CGI tool. Set the
LD\_LIBRARY\_PATH as you set it for the web server cgi directory,
and try test.sh again.

\noindent You can also try this to see more of the web server response:

\begin{lstlisting}[keywordstyle=\sffamily]
telnet my_web_server_host 80
GET /archive/cgi/ArchiveDataServer.cgi HTTP/1.0<RETURN>
<RETURN>
\end{lstlisting}

\noindent You should see an error message about ``XML-RPC Fault: Expected HTTP method POST'',
indicating that the data server was launched correctly and
was looking for a proper XML-RPC request.
If you see only pages of binary-looking garbage, your web server
doesn't recognize ArchiveDataServer.cgi as a CGI program,
and instead of running it, you get a copy of it.

\subsection{Standalone Data Server} % ---------------------------------------------
The preferred deployment of the network data server is within a
standard web server as described in the previous sections.
For experiments, however, there is a way to use a version of the 
data server which combines the simple ``Abyss''
web server and the network data server into one standalone program.

\noindent Advantages:
\begin{itemize}
\item Easier to configure than a standard web server. No
  serverconfig.xml, no serverconfig.dtd, no CGI setup trouble. When running
  ``ArchiveDataServerStandalone'', you will see right away if e.g.\ a
  shared library is missing, while the ``ArchiveDataServer''
  CGI-plugin to a web server would simply not run for reasons harder
  to diagnose.
\item Ordinary users beyond ``root'' or ``Administrator''
  can run the standalone data server.
\end{itemize}

\noindent Disadvantages:
\begin{itemize}
\item The Abyss HTTPD still requires some configuration.
\item Compared to e.g.\ the Apache web server, there is much less
  control over who can access the data.
\item Currently limited to serving a single archive per running
  instance of the standalone data server.
  The 'key' of that archive is fixed to '1' and  the 'name' to 'Archive'.
  Under a standard web server, one can run more than one
  ArchiveDataServer and configure each one to serve multiple archives
  with selected key and name.
\end{itemize}

\subsection{Running ArchiveDataServerStandalone} % ---------------------------------------------
After successful compilation in ChannelArchiver/XMLRPCServer, you will
have a program ``ArchiveDataServerStandalone''.

\begin{enumerate}
\item For starters, you can test in the ``ChannelArchiver/DemoData''
      directory, but typically you would copy all ``abys*'' files and
      directories from there into the directory where you intend to
      run the standalone data server.
\item Edit ``abyss.conf'' to suit your needs. Most users might only
      have to adjust the ``Port'' option, which defaults to 8080,
      and the ``ServerRoot'' variable.
\item Run the data server with the abys config file and the path of
      the archive's index. When inside DemoData, this would be an example:
\begin{lstlisting}[keywordstyle=\sffamily]
> ArchiveDataServerStandalone abyss.conf index 
 ArchiveDataServerStandalone Running
 Unless your 'abyss.conf' selects
 a different port number,
 the data should now be available via the XML-RPC URL
    http://localhost:8080/RPC2
\end{lstlisting}
\end{enumerate}

\noindent The data is now accessible via XML-RPC by
pointing the ArchiveDataClient.pl test tool or the Java
ArchiveViewer to the URL
\begin{lstlisting}[keywordstyle=\sffamily]
     http://<hostname>:<port>/RPC2
\end{lstlisting}
\noindent Example:
\begin{lstlisting}[keywordstyle=\sffamily]
    http://localhost:8080/RPC2
\end{lstlisting}

