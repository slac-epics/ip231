\chapter{Indices} \label{chap:indices}
The archiver supports two types of indices:
\begin{itemize}
\item ``binary'':\\
      This is the type of index that the engine creates for its data
      files. The ArchiveIndexTool, described in \ref{sec:indextool},
      allows the creation of new indices which combine data from
      existing indices.

      A \INDEX{binary index} contains information about all the data block for
      each channel. It offers the best retrieval performance.  On the
      downside, the creation of a binary index can take time, and the
      size of an index file is limited to 2 GB. When the sub-archives
      which are accessed via the index change, the index needs to be
      updated or rebuild.
\item ``list'':\\
      This type of index simply lists the sub-archives which one would
      like to access. Its format is the same as the indexconfig.xml
      used by the ArchiveIndexTool to create a binary index.
      When retrieving data, the first sub-archive in the list is
      used. When the channel is not found, the next sub-archive is
      used and so on.

      A \INDEX{list index} takes virtually no time to create, and also
      requires little disk space. The retrieval is not too bad when
      only a few sub-archives are listed. It works very well for
      maybe a dozen entries, but of course degrades with
      the number of entries.

      It will also fail when the sub-archives contain the same
      channels but for different time ranges.
\end{itemize}

\noindent One should try to use binary indices as long as the size and update times
are acceptable.
Not available but clearly needed is a third index type which contains
the channel names and time ranges of the sub-archives, without growing
to the full binary index size which references each data block.

