% MEDM operators manual
%
\documentstyle [12pt] {article}
\title{
{\huge MEDM\\{\Large Motif-based Editor/Display Manager}\\
\vspace{12pt}Operator's Manual}
\vspace{24pt}
\author{Mark D. Anderson\\
mdanderson$@$anl.gov\\
Argonne National Laboratory}
}

\begin{document}
\maketitle

\begin{abstract}
\bf
\noindent Some operational hints about the use of the Motif-based
editor/display manager (medm) are described here for the user.  This is not
intended to be a full description of the features of MEDM.
\end{abstract}

\section{Introduction}

MEDM is an interactive display screen creation, editing and execution tool,
which allows users to create dynamic displays in the EPICS environment.
Channel Access is used as the communication protocol to objects (channels) in
the IOC namespace, and channels in that name space can have objects
`attached' to them such that the channel (process variable) behavior results
in dynamic visual behaviors.\\


\section{X11 requirements}

Motif and other internationalized applications must have proper locale
and I18N environments setup.  This includes, for X11R5, /usr/lib/X11/nls/
or the XNLSPATH environment variable set to the appropriate directory
(usually .../lib/X11/nls under the X11 installed tree).  Failure to do
this will result in crashes of such applications!\\

\noindent For utilization of all of MEDM's features, it is highly recommended
that the X server used be X11R5 (or later) based.  Note that SUN's OpenWindows
XNews server is {\bf not} a supported X server! \footnote{Most MIT X and
Motif applications rely upon proper X server behavior
with respect to ordered event delivery, et cetera. SUN's XNews server has known
bugs in this regard (witness SUN's bug list for the server) and is being
replaced by an MIT code-base X11R5 server (in OpenWindows 3.3, as well
as the COSE Common Desktop Environment).  Once CDE is available for SUNs, this
problem will no longer be an issue.}


\section{Command Line}

MEDM is invoked as "medm" on the command line with optional arguments
for controlling its operational modes.  The standard X resource command
line options (such as -display xyz:0, etc) are also valid options.

\noindent The structure of the command line is:\\

\begin{tabular}{ll}
{\bf \% medm} & {\bf [-x $|$ -e] [-local $|$ -cleanup] [-cmap]
[-displayFont $<font>$]}\\
	&  {\bf [-macro $"<name>=<value>,<name>=<value>..."$]}\\
	& {\bf [...X based options...] [display file names]} \\
\end{tabular}

\vspace{12pt}
\begin{tabular}{ll}
{\bf -x}	& startup medm in execute-only (or ``dm'') mode\\
{\bf -e}	& startup medm in edit/execute (or ``edd/dm'')\\
		& mode (the default)\\
{\bf -local}	& don't participate in remote display protocol\\
		& (the default is to participate and dispatch\\
		& display requests to a remote MEDM if possible)\\
{\bf -cleanup}	& (seldom needed) support remote display protocol,\\
		& but ignore existing MEDM and "take possession" of\\
		& remote display responsibilities\\
{\bf -cmap}	& use private colormap to circumvent unallocable\\
		& colors in default colormap (this may cause colormap\\
		&  flashing)\\
{\bf -displayFont $<font>$} & select aliased or scalable fonts (see {\bf Fonts}\\
		& section below) in default colormap (this may cause\\
		& colormap flashing)\\
{\bf -macro $"<n=v>,..."$}    &  apply macro substitution as specified in the\\
		& name=value (n=v)... string to all command-line\\
		& specified display list files (this option requires\\
		& -x also)\\
{\bf $*$.adl}	& a list of ascii display list files\\
\\
\end{tabular}

\vspace{12pt}
\noindent For example:\\

\begin{tabular}{ll}
medm			& starts up in default (edit) mode\\
medm -e xyz.adl		& starts up editing one display \\
medm -local		& starts up in default (edit) mode\\
			& starts up new local executable\\
			& doesn't participate in remote\\
			& protocols\\
medm -x abc.adl def.adl	& starts up executing two displays,\\
			& and does take advantage of or support\\
			& display protocol\\
medm -cmap		& starts up using a private colormap;\\
			& this is useful when run on a display\\
			& with other color-greedy applications\\
medm -displayFont scalable & starts up using default scalable\\
			& fonts (medm default is aliased)\\
medm -x -macro "a=b,c=d" t1.adl & starts up in execute mode,\\
			& performing macro substitution on all\\
			& occurrences of \$(a) and \$(c) in\\
			& display file t1.adl\\
 
\end{tabular}

\vspace{12pt}
\noindent N.B. The usual X/Xt oriented command line arguments are supported
also, by default


\section{Composite Product}

{\bf MEDM performs the functions of both dm and edd; i.e., it performs
both display creation/editing and display execution functions.}\\

\noindent Once medm is running, displays are opened via selecting
File$\rightarrow$Open...  from the main menu and selecting the desired ADL
($*$.adl) file from
the dialog box.  The file is opened in either Edit or Execute mode, based on
the selected mode in the main window.  These toggle buttons allow the user
to alternately go between Edit and Execute for all opened displays.
When in Execute mode, most Edit functions are not available for selection,
and normal Execute semantics are provided.

\subsection{Normal Motif Operation}

MEDM conforms to the Motif Style Guide.  Hence standard mnemonics and
accelerators are available for interface navigation. 


\subsection{Operational Aspects of the Interface}

\subsubsection{Mode Buttons}

MEDM has two modes of operation, {\bf Edit} and {\bf Execute};
the edit mode is the mode in which screens are created; execute mode is the
mode in which screens are activated and attached to underlying channels in IOCs.
Thus, in execute-mode, screen objects exhibit behavior based on the state of
their associated channels.  MEDM's mode is reflected and changed by
interaction with the Edit and Execute buttons (in a Mode radio box) in
MEDM's main window.

\subsubsection{Menu Bar}

This application has five pulldown menus in the main window:
File, Edit, View, Palettes, and Help.

\subsubsection{File}

The File pulldown menu controls file-based operations, and overall program
execution.\\

\begin{tabular}{ll}
\underline{New} & create a new display\\
\underline{Open...} & open an existing display\\
\underline{Save} & save an opened display\\
\underline{Save As...} & save an opened display as a (potentially new) file\\
\underline{Close} & close an opened display\\
\underline{Print...} & print an opened display\\
\underline{Exit} & exit MEDM\\
\end{tabular}


\subsubsection{Edit}

The Edit pulldown menu controls editing operations.\\

\begin{tabular}{ll}
\underline{Cut}  & cut selected objects from the display into the clipboard\\
\underline{Copy}  & copy selected objects from display into clipboard\\
\underline{Paste}  & copy objects in clipboard into display\\
\underline{Delete}  & delete selected objects from display \\
\underline{Raise}  & place selected objects in front of other objects\\
\underline{Lower}  & place selected objects behind other objects\\
\underline{Group}  & group selected objects into single composite object\\
\underline{Ungroup}  & return a group to its constituent objects\\
\underline{Align$\rightarrow$}  & align selected objects various ways\\
\underline{Unselect}  & unselect any selected objects\\
\underline{Select All}  & select all objects in the display\\
\end{tabular}

\subsubsection{View}

View is currently not implemented.

\subsubsection{Palettes}

There are three palettes which can be popped up, which are described later in
this document:\\

\begin{tabular}{ll}
\underline{Object...} & object palette which allows object creation\\
\underline{Resource...} & resource palette which displays and allows editing of object resources\\
\underline{Color...} & color palette which allows color assignments to objects\\
\end{tabular}

\subsubsection{Help}

Help is currently unfinished; `On Version...' is implemented and displays
the version information about the instance of MEDM running.

\subsection{Object Palette}

The object palette presents the available objects with which a user can 
construct a display.  These are:\\

\begin{tabular}{ll}
Graphics: & \underline{Rectangle}, \underline{Oval}, \underline{Arc},
	  \underline{Text}, \underline{Line},\\
	  & \underline{Image}, \underline{Polygon}, \underline{Polyline}\\
\\
Monitors: & \underline{Meter}, \underline{Bar}, \underline{Indicator},
\underline{Cartesian Plot}\\
	  &  \underline{Strip Chart}, \underline{Text Update}\\
\\
Controllers: & \underline{Choice Button}, \underline{Menu},
\underline{Text Controller}, \\
	 & \underline{Valuator}, \underline{Shell Command},\\
	 &  \underline{Message Button}, \underline{Related Display}\\
\\
Misc: & \underline{Select... (selection mode cursor)}\\
\end{tabular}


\subsection{Resource Palette}

Each of the above objects has a set of resources which are editable
by the user; these resources describe the appearance and behavior of
the object.  Some common resources include: X, Y, Width, Height,
Foreground, etc.  Other less universal resources include Fill Style,
Color Modes, Visibility, Line Width, Channel, Readback Channel (for
monitor objects), Control Channel (for controller objects), etc. \\

\noindent See MEDM's resource palette for the resources associated with each
object type.

\subsection{Color Palette}

The color palette is popped up whenever a color selection for an object
is required (e.g., when the Foreground or Background button is pressed
in the resource palette).  The color is assigned by selecting (pressing)
the desired color in the color palette.  The color palette remains
visible after the selection for the user to assign and inspect
many colors in succession if desired.


\section{Fonts}

MEDM uses several methods for font specification.  The flag\\

\begin{tabular}{ll}
$-$displayFont & $<font>$ \\
	     & with $<font>$ one of \{alias, scalable, $<user-specified>$\}\\
\end{tabular}
\vspace{12pt}

\noindent allows users to select the default (aliased names), default scalable,
or user-specified scalable fonts.  The default is $-$displayFont alias.

\subsection{Default, Fixed Fonts}

For default, fixed fonts, MEDM uses ``logical font names'' internally, to deal
with server font variations.  Consequently, font aliases are necessary for the
various servers on which medm will run.  For the SUNs, the file
fonts.alias.sun should be copied into a common (or user specific) font area
as fonts.alias, and the font path for the server set to include that
directory.\\

\noindent The default (no $-$displayFont specified) is the same as:\\

\% medm $-$displayFont alias\\

\noindent For example, a user may create an Medm directory, copy medm and
fonts.alias.sun into that area, and then move (cd) to that directory.  The
user may then\\

\% mv fonts.alias.sun fonts.alias\\

\% xset $+$fp \$cwd/  \hspace{24pt}(or other full path specification terminated by /)\\

\noindent To verify that the fonts are included, do a\\

\% xset $-$q\\

\noindent to see that the font path is setup appropriately, and do a\\

\% xlsfonts $|$ grep widgetDM\\

\noindent to verify that the logical font names are resolved.\\


\subsection{Default, Scalable Fonts}

The user can invoke MEDM with the   $-$displayFont scalable    option.
MEDM then uses the default Speedo outline font (bitstream) supplied
by the X11R5 font server.  Users should add a font server to their
font path via:\\

\% xset $+$fp tcp/$<hostname>$:$<portnumber>$\\

\noindent For example:\\

\% xset $+$fp tcp/phebos:7000\\

\% medm $-$displayFont scalable\\


\subsection{User-Specified, Scalable Fonts}

The user can invoke MEDM with the   $-$displayFont $<font>$    option.
MEDM then uses the specified font supplied by the X11R5 font server.
This font should be an XLFD name (all 14 hyphens, other fields can be
wildcarded though) and scalable (i.e., point and pixel size = 0).
Users should add a font server to their font path as above.

\noindent For example:\\

\% medm $-$displayFont $-$bitstream-courier-medium-r-normal--0-0-0-0-m-0-iso8859-1\\



\section{Accelerators and Translations}

There are several built-in accelerators or translations for mouse and
keyboard events in the Motif interface built with the Motif-based display
manager.\\

\subsection{Main Display Popup Menu}

The main display popup menu (with Print and Close options for the current
display) is popped-up by depressing MB3 in nearly any location on the display
window. One of these options is selected by releasing MB3 when the desired
item is under the cursor and ``rasied''.


\subsection{Valuator/Scale}

The valuator/scale object has several modes of operation, which implement
fine/coarse sensitivity, as well as direct value selection.\\

\noindent Dragging the valuator/scale with MB1 depressed moves and transmits
values proportional to the range of the valuator scaled into the width of the
valuator (usually  $>=$ 1\% for this application, depending upon the selected
width of the valuator). Clicking MB1 on either side of the valuator selector
moves and transmits a value increment or decrement of specified precision
(for this application) of the valuator object.  Clicking $<$Ctrl$>$-MB1 on
either side of the valuator selector moves and transmits a value increment
or decrement of 10x specified precision of the valuator object.\\

\noindent In addition to the normal mouse-activated mode of usage for the
valuator/scale, the arrow keys and shift key take on functions as well.
Fine-grained Increment 
and Decrement are accomplished with the up/down or left/right arrow keys, 
depending upon the orientation of the valuator/scale.  The slider will move 
in the direction ``pointed to'' by the arrow key being pressed (as expected), 
when input focus is set to that widget.  For instance, $<$right-arrow$>$ will 
move the slider by $+$1 precision unit when the orientation is HORIZONTAL and 
processingDirection is MAX\_ON\_RIGHT.\\

\noindent In conjunction with increment/decrement, the $<$Ctrl$>$ key can be
used to specify a coarse-grained increment/decrement.  Hence,
$<$Ctrl$>$$<$up-arrow$>$,
for instance, will move the slider by $+$10 precision units when the orientation
is VERTICAL and processingDirection is MAX\_ON\_TOP.\\

\noindent The specified precision for motion of the valuator is specified in a
dialog box which is popped up by depressing MB3 in the valuator/scale. A series
of toggle buttons with values of log10(precision) are selectable, with
the current precision indicated by the depressed toggle.
Also, for the highest precision modification of associated process variables,
the keyboard entry dialog box can be used to specify a direct-entry
value to be written to the valuator and channel.\\



\subsection{Text Entry/Text Field}

Several edit modes are supported for the text entry or text field widget.
In addition to point-click positioning, the left and right arrow keys
allow the cursor to be positioned anywhere in the field. Similarly, backspace
and delete allow characters to be removed from the string/field.  Carriage
Return ($<$CR$>$) sends the value;  additionally, leaving the widget also 
transmits the current value back to the application.\\

\section{Related Displays}

As in the Xlib-based display manager, the environment variable
EPICS\_DISPLAY\_PATH should contain the directory/directories where
related displays (for a given display) can be located. This can
be a single directory name, or a colon-separated list of directories,
including {\bf .}, {\bf ..}, etc.  For example:

\% setenv EPICS\_DISPLAY\_PATH .:..:/usr/tmp/displays:/usr/tmp/adl\\




\section{Display Editing}

Based on the selected button in the object palette, the editor can be
in CREATE or SELECT mode.  These modes determine the semantics of
button presses in an active display.

\newpage
\noindent When the select button is depressed in the object palette, the
editor is in SELECT mode and the following semantics apply:\\

\begin{tabular}{ll}
MB1: & select an object (or objects) on the screen and\\
     & highlight. the resource palette is updated to\\
     & reflect the selected item's internal data.\\
     & a drag operation under MB1 selects a group of\\
     & objects on the screen (including those objects\\
     & wholly bounded by the selection rectangle).\\
\\
Shift-MB1: & multiple-select.  a set of objects are selected \\
           & for operations (such as grouping).\\
\\
MB2: & selected object(s) are moved while MB2 is depressed \\
     & and deposited on button release.  if no objects are\\
     & currently selected, the object under the cursor when\\
     & MB2 is depressed in made the current object for \\
     & moving. to cancel the effect of the current move,\\
     & the cursor may be dragged off the current display\\
     & window and the button released.  this cancels the\\
     & effect of the move.\\
\\
Ctrl-MB2: & selected object(s) are resized while MB2 is depressed.\\
	  & if no objects are currently selected, the object under\\
	  & the cursor when MB2 is depressed in made the current\\
	  & object for resizing. to cancel the effect of the current\\
	  & resize, the cursor may be dragged off the current display\\
	  & window and the button released.  this cancels the\\
	  & effect of the resize.\\
	  & N.B. this mechanism performs ABSOLUTE resizing, in which\\
	  & all objects are extended in width and height by the\\
	  & magnitude of the x and y mouse motion.  for PROPORTIONAL\\
	  & resizing in which selected objects resize consistently,\\
	  & the object must first be grouped.\\
\\
MB3: & popup applicable menus.\\
\end{tabular}

\newpage
\noindent When an object (e.g. rectangle) button is depressed in the object
palette, the editor is in CREATE mode and the following semantics apply:\\

\begin{tabular}{ll}
MB1: & an object of current type is created, starting at\\
     & the origin of button press, of size determined by\\
     & button release (a bounding rectangle is rubberbanded).\\
     & the resource palette is updated to reflect the object's\\
     & internal data.\\
\\
MB2: & (as in SELECT mode above)\\
     & selected object(s) are moved while MB2 is depressed\\
     & and deposited on button release.  if no objects are\\
     & currently selected, the object under the cursor when\\
     & MB2 is depressed in made the current object for \\
     & moving. to cancel the effect of the current move,\\
     & the cursor may be dragged off the current display\\
     & window and the button released.  this cancels the\\
     & effect of the move.\\
\\
Ctrl-MB2: & (as in SELECT mode above)\\
	  & selected object(s) are resized while MB2 is depressed.\\
	  & if no objects are currently selected, the object under\\
	  & the cursor when MB2 is depressed in made the current\\
	  & object for resizing. to cancel the effect of the current\\
	  & resize, the cursor may be dragged off the current display\\
	  & window and the button released.  this cancels the\\
	  & effect of the resize.\\
	  & N.B. this mechanism performs ABSOLUTE resizing, in which\\
	  & all objects are extended in width and height by the \\
	  & magnitue of the x and y mouse motion.  for PROPORTIONAL\\
	  & resizing in which selected objects resize consistently,\\
	  & the object must first be grouped.\\
\\
MB3: & (as in SELECT mode above)\\
     & popup applicable menus.\\
\end{tabular}\\


\noindent For fine-tuning placement of objects when editing, use the arrow keys
(up/down/left/right) to move selected objects in the display one pixel
at a time in the direction of the arrow.  Note that many widgets trap
input events, therefore  the cursor should not be positioned over a
widget when this motion is being requested.  Input focus must be
on the display (or its top level shell) for this mechanism to work.\\


\section{New features}

\subsection{Image}

Images can be imported for inclusion in a display.  The image
icon (camera) can be selected, and the location and size of
the desired image then selected by depressing and dragging
MB1 in the display.  A file selection box then prompts the
user for the display file to be incorporated.  At the present
time, GIF is the only supported format, and files must have
the ``.gif'' suffix.\\

\subsection{Shell Command}

Shell commands can be incorporated into a display
by selection of the ``shell'' (exclamation point) icon,
followed by MB1 depression
and dragging in the display.  The labels and commands (and
optional arguments) are then specifiable in the ``Label/Cmd/Args''
dialog box.  Note:  for prompted input, a question mark ``?'' in
the command or args field instructs the program to popup a
dialog to allow the user to complete the command string before
execution.  Also note:  the shell command blocks by default
upon execution.  To support asynchronous behavior, the command
or argument should be followed by the ampersand ``\&'' to instruct
the system to run the command in the background.  All stdout/stderr
output, at this time, is still tied to the controller terminal
window.  Also note that shell command and argument strings can
be macro-substituted via the \$(name) construct (see related
display description in this document).\\

\subsection{Cartesian Plot}

Cartesian plot data will utilize up to 2 Y axes for
display.  Trace 0 will use the left Y axis, traces 1-7 (if
applicable) will use the right Y axis.  Hence, to plot power
vs. thermocouple values, trace 0 could be power, and traces 1-7
could be the thermocouples (since these will probably all have
the same scale, sharing the Y axis is preferable).\\

\noindent Cartesian Plots can be made of scalars, scalar vs. scalar,
vector, vector vs. scalar, and vector vs. vector.  For ``incomplete''
data such as vector (vs. nothing) the other independent variable
will be filled in with an index (element position number).
Vector vs. scalar allows users to simulate display of error bars
by displaying a set (vector) of data at a point (similarly, scalar
vs. scalar with one scalar value fixed and the other varying, with
a count greater than 1 and erase oldest ON can show ``history'' of
values at a point).  Vector vs. vector supports ``ordered pair plots''
where users can supply an X vector of data and a Y vector of data,
with the resultant plot being (x[0],y[0]), (x[1],y[1]), etc.\\

\noindent Cartesian Plots update internal data as any constituent channels
update;  the visual state of the plots can be updated either as a result
of any channel updates, or as the result of a {\bf trigger channel} update.
Users can specify a channel as a trigger channel and tie visual updates
of the plot to the update rate of the trigger channel.  This throttles
visual updates to predictable and possibly more optimal/efficient rates.\\

\subsection{Macro Substitution} 

MEDM now supports macro substitution (including nested,
or parent-to-child substitution) for related displays.  \\

\noindent Related displays can be called with an arguments string of the form
``name=value [,name=value,...]''.  The related display then substitutes
in its space all occurrences of ``\$(name)'' with ``value''.\\

\noindent A parent related display can pass to it's child related display a
value in its space by passing an argument string of the form
``name=\$(name)XYZ'', etc.\\

\noindent Shell command and argument strings are also substituted in related
displays, and referenced similarly via the \$(name) construct.\\

\subsection{Polyline and Polygon Support}

Polylines and polygons are available from
the object palette, in both ``constrained'' and ``unconstrained''
form.  Unconstrained drawing allows the user to click (MB1) and
generate vertices for a polyline or polygon arbitrarily.
SHIFT-MB1 allows the user to generate vertices which are multiples
of 45 degrees from the previous point (this allows easy horizontal
or vertical line generation, for instance).\\

\noindent Note that polygons in fill mode ``outline'' are similar to polylines,
except that polygons will close the figure and polylines do not.
Also note that Vertex editing is now available for Line, Polyline
and Polygon objects.  Selection of vertices of selected objects
via MB1-press and subsequent drag until MB1-release allows vertices
to be moved.\\



\section{Drag-And-Drop}

MEDM, in execute mode, allows via the drag-and-drop mechanism (MB2),
the deposition of object process variable names onto other
objects or programs.  For example, depressing MB2 on a controller
(e.g., valuator) will retrieve the underlying channel's name, and allow
it to be dropped onto other objects (e.g., text fields in the same or other
Motif applications) or other programs (e.g., KM - the knob manager program,
which for instance, can then assign that channel to the specified knob
and initiate physical control).\\

\noindent This DND functionality has been extended to include {\bf all}
monitor-able objects (i.e., objects which have an associated process
variable): controllers, monitors, and dynamic graphic objects.  Note
that for Cartesian Plots and Strip Charts, vectors or matrices of
channel names are "dragged".  Other DND clients must gain knowledge of
these vectors of data for the vector DND protocol to be useful to them.
Note that the channel names are rendered according to the alarm color
of the underlying channel; thus the DND operation conveys underlying
channel name and channel alarm severity, in the usual manner (R = major
alarm, Y = minor alarm, G = no alarm, W = valid/invalid (e.g. connection
lost) alarm).\\



\section{``Smart Startup''}

In order to amortize the cost of Xt Intrinsics and Motif initialization
over many displays, and to offer an efficient way to start up MEDM
displays ``after-the-fact'', a remote MEDM request protocol has
been established.  \\

\noindent MEDM by default now looks for other MEDM's running (on ANY machine)
which have the same display device as the requested display.  If one is found,
the display request is forwarded to the remote MEDM for processing.  This
saves startup time of several MEDM processes which are all going to the
same display.\\

\noindent To override this behavior, use the ``-local'' parameter on the command
line.  MEDM invoked with -local (in either edit or execute mode) will
not participate in the ``smart-startup'' protocol and act independently
of other MEDMs.  (The default is not -local, i.e., DO participate in
smart-startup protocol).\\

\noindent N.B. - this request protocol does not work when the two machines
running MEDM do not share a file namespace.  Also note that this obviates the
MEDM startup library and application messaging scheme which has been 
discussed.\\

\noindent MEDM should always be stopped via the File$\rightarrow$Exit
menu selection.
If MEDM is killed via a SIGKILL or SIGSTOP signal, some X window
property cleanup may be required.  If MEDM is invoked, makes a remote
display request and returns (very quickly) but no display is ever
created, then start MEDM in ``cleanup'' mode:\\

\% medm -cleanup ...\\

\noindent This starts up MEDM in a mode which ignores remote MEDMs, and does
the proper X property cleanup on exit.\\

\noindent Alternatively, to effect the same ``cleanup'' as above, the
X properties can be deleted from the X server manually:\\

\% xprop -root $|$ grep MEDM \hspace{24pt} \# see if an MEDM property is stored\\

\noindent If there is one, then do the following:\\

\% xprop -root -remove $<property>$\\

% note this uses \[ \] for "displaymath" env. rather than simple math env.
\noindent where \[ <property>  \hspace{12pt} \in \left\{ \begin{array}{ll}
		MEDM\_EDIT\_FIXED, & MEDM\_EDIT\_SCALABLE\\
		MEDM\_EXEC\_FIXED, & MEDM\_EXEC\_SCALABLE\end{array} \right\}
\] \\

\noindent depending on which mode MEDM you are requesting (medm -e or -x)
and whether fixed (aliased) or scalable fonts are used (-displayFont alias
or scalable).\\


\section{Aspect Ratio Preserving Resize}

MEDM in execute mode allows run-time resizing of displays.  Since the user
is totally free to pick sizes to his liking, the aspect ratio of the
original display can be severely distorted.  Users wishing to preserve
the aspect ratio of a display at run-time can do a SHIFT-modified resize.
To do this, the user invokes the resize of the display (window) as appropriate
for the window manager in use, with the Shift key depressed at the time of
completion of the resize operation.\\

\noindent Note that the resizing is not constrained at the time of resize, but
upon completion the display will be resized to (``snap to'') the appropriate
sizes of the width and height dimensions.  Both reducing and enlarging of
the display is supported.\\



\end{document}
